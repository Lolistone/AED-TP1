\documentclass[10pt,a4paper]{article}

\input{AEDmacros}
\usepackage{caratula} % Version modificada para usar las macros de algo1 de ~> https://github.com/bcardiff/dc-tex


\titulo{Trabajo Práctico 1}
\subtitulo{Especificación y Weakest Precondition}

\fecha{\today}

\materia{Algoritmos y Estructuras de Datos}
\grupo{swifties}

\integrante{Apellido, Nombre1}{001/01}{email1@dominio.com}
\integrante{Apellido, Nombre2}{002/01}{email2@dominio.com}
\integrante{Apellido, Nombre3}{003/01}{email3@dominio.com}
\integrante{Apellido, Nombre4}{004/01}{email4@dominio.com}
% Pongan cuantos integrantes quieran

% Declaramos donde van a estar las figuras
% No es obligatorio, pero suele ser comodo
\graphicspath{{../static/}}

\begin{document}

\maketitle

\section{Especificación}

\noindent Algunas consideraciones que tuvimos al especificar:
\begin{enumerate} \setlength\itemsep{0cm}
	\item Un escrutinio tiene al menos dos partidos, y en la ultima posicion estan los votos en blanco.
	\item Tanto en el escrutinio como en la matriz DHont no hay repetidos.
	\item Los escrutinios en hayFraude tienen la misma longitud.
	\item Los votos en el escrutinio y los cocientes de la matriz son numeros mayores o iguales a 0.
	\item En la matriz DHont los partidos que no pasaron el umbral estaran representados como una lista vacia.
	\item En las listas de los partidos representamos a hombres con el 1 y mujeres con el 2.
\end{enumerate}
Los predicados y auxiliares utilizados están definidos al final de la sección. \\

\subsection{Ejercicio 1}

\begin{proc}{hayBallotage}{\In escrutinio : \TLista{\ent}}{Bool}
	%    \modifica{parametro1, parametro2,..}
	\requiere{escrutinioValido(escrutinio)}
	\asegura{res = True \longleftrightarrow \neg ((\paraTodo[unalinea]{j,k}{\ent}{0 \leq j,k < |escrutinio| - 1  \yLuego porcentaje(j, escrutinio) > 40  	\implicaLuego \\
		   	porcentaje(j, escrutinio) - porcentaje(k, escrutinio) > 10}) \lor \\
		  	\existe[unalinea]{i}{\ent}{\paraTodo[unalinea]{j}{\ent}{0 \leq i,j < |escrutinio| - 1 \yLuego porcentaje(i, escrutinio)  > 45 \land i \neq j  \implicaLuego \\ porcentaje(j, escrutinio) \leq 45}})}
\end{proc}

\subsection{Ejercicio 2}

\begin{proc}{hayFraude}{\In escrutinio\_presidencial : \TLista{\ent}, \In escrutinio\_senadores : \TLista{\ent}, \In escrutinio\_diputados : \TLista{\ent}}{Bool}
	%    \modifica{parametro1, parametro2,..}
	\requiere{|escrutinio\_presidencial| = |escrutinio\_senadores| \land |escrutinio\_senadores| = |escrutinio\_diputados| \land \\
	escrutinioValido(escrutinio\_presidencial) \land escrutinioValido(escrutinio\_senadores) \\ \land escrutinioValido(escrutinio\_presidencial)}
	\asegura{res = True \longleftrightarrow totalVotos(escrutinio\_presidencial) =  totalVotos(escrutinio\_senador) \land  \\  totalVotos(escrutinio\_senadores) =  totalVotos(escrutinio\_diputados)}
\end{proc}

\subsection{Ejercicio 3}

\begin{proc}{obtenerSenadoresEnProvincia}{\In escrutinio : \TLista{\ent}}{$\ent \times \ent$}
	%    \modifica{parametro1, parametro2,..}
	\requiere{escrutinioValido(escrutinio)}
	\asegura{(0 \leq res_{0}, res_{1} < |escrutinio| - 1) \yLuego \paraTodo[unalinea]{i}{\ent}{0 \leq i < |escrutinio| - 1 \land i \neq res_{0} \land 
	i \neq res_{1} \implicaLuego \\ escrutinio[i] < escrutinio[res_{1}] < escrutinio[res_{0}]}} 
\end{proc}

\subsection{Ejercicio 4}

\begin{proc}{calcularDHontEnProvincia}{\In cant\_bancas : \ent, \In escrutinio : \TLista{\ent}}{\TLista{\TLista{\ent}}}
	%    \modifica{parametro1, parametro2,..}
	\requiere{cant\_bancas > 0 \land escrutinioValido(escrutinio)}
	\asegura{|res| = |escrutinio| - 1 \yLuego \\
	\paraTodo[unalinea]{i}{\ent}{0 \leq i < |res| \yLuego superaUmbral(escrutinio[i]) \implicaLuego |res[i]| =  cant\_bancas} \land \\
	\paraTodo[unalinea]{i}{\ent}{0 \leq i < |res| \yLuego \neg(superaUmbral(escrutinio[i])) \implicaLuego |res[i]| =  0} \land \\
	\paraTodo[unalinea]{i,j}{\ent}{0 \leq i < |res| \yLuego 0 \leq j < |res[i]| \implicaLuego res[i][j] =  escrutinio[i]/(j + 1)}}
\end{proc}

\subsection{Ejercicio 5}

\begin{proc}{obtenerDiputadosEnProvincia}{\In cant\_bancas : \ent, \In escrutinio : \TLista{\ent} \In DHont : \TLista{\TLista{\ent}}}{\TLista{\ent}}
	%    \modifica{parametro1, parametro2,..}
	\requiere{cant\_bancas > 0 \yLuego dHontValida(DHont, cant\_bancas, escrutinio)}
	\asegura{|res| = |DHont| \yLuego \\ \paraTodo[unalinea]{i}{\ent}{0 \leq i < |res| \implicaLuego res[i] = \sum\limits_{j=0}^{|DHont[i]|-1} \IfThenElse{cant\_bancas > \#cocientesMayores(DHont[i][j], DHont)}{1}{0}}}
	 
\end{proc}

\subsection{Ejercicio 6}

\vspace{0.1cm}

\begin{proc}{validarListasDiputadosEnProvincia}{\In cant\_bancas : \ent, \In listas : \TLista{\TLista{dni : \ent \times genero : \ent}}}{Bool}
	%    \modifica{parametro1, parametro2,..}
	\requiere{cant\_bancas > 0 \land generosValidos(listas)}
	\asegura{res = True \longleftrightarrow \\
\paraTodo[unalinea]{i}{\ent}{0 \leq i < |listas| \implicaLuego  |listas[i]|= cant\_bancas} \land \\ 
\paraTodo[unalinea]{i,j}{\ent}{0 \leq i < |listas| \yLuego 0 \leq j < |listas[i]| - 1 \yLuego listas[i][j]_{1} = 1 \implicaLuego  listas[i][j+1]_{1} = 2 } \land \\  \paraTodo[unalinea]{i,j}{\ent}{0 \leq i < |listas| \yLuego 0 \leq j < |listas[i]| - 1 \yLuego listas[i][j]_{1} = 2 \implicaLuego  listas[i][j+1]_{1} = 1 }}
\end{proc}

\subsection{Predicados y Auxiliares}

\aux{porcentaje}{\In partido : {\ent}, \In escrutinio : \TLista{\ent}}{\ent}{(escrutinio[partido] * 100) / totalVotos(escrutinio)}
\aux{\#cocientesMayores}{\In i : \ent, \In j : \ent, \In DHont : \TLista{\TLista{\ent}}}{\ent}{\\ \sum\limits_{r=0}^{|DHont[i]|-1} \sum\limits_{k=0}^{|DHont|-1} \IfThenElse{DHont[k][r] > DHont[i][j]}{1}{0}}
\aux{totalVotos}{\In escrutinio : \TLista{\ent}}{\ent}{\sum\limits_{i=0}^{|escrutinio|-1} escrutinio[i]}
\pred{noHayEmpate}{\In escrutinio : \TLista{\ent}}{\paraTodo[unalinea]{i,j}{\ent}{0 \leq i,j < |escrutinio| - 1  \land i \neq j  \implicaLuego escrutinio[i] \neq escrutinio[j] }}
\pred{votosValidos}{\In escrutinio : \TLista{\ent}}{\paraTodo[unalinea]{i}{\ent}{0 \leq i < |escrutinio| \implicaLuego escrutinio[i] \geq 0 }}
\pred{escrutinioValido}{\In escrutinio : \TLista{\ent}}{|escrutinio| > 2 \land noHayEmpate(escrutinio) \land votosValidos(escrutinio)}
\pred{cocientesDistintos}{\In DHont : \TLista{\TLista{\ent}}}{\paraTodo[unalinea]{i,j,k,r}{\ent}{0 \leq i,k < |DHont| \yLuego 0 \leq j, r < |DHont[i]| \land
|DHont[i]| = |DHont[k]| \land (i \neq k \lor j \neq r) \implicaLuego 
DHont[i][j] \neq DHont[k][r]}}
\pred{cocientesPositivos}{\In DHont : \TLista{\TLista{\ent}}}{\paraTodo[unalinea]{i,j}{\ent}{0 \leq i < |DHont| \yLuego 0 \leq j < |DHont[i]| \implicaLuego DHont[i][j] \geq 0}}
\pred{filasValidas}{\In DHont : \TLista{\TLista{\ent}}, \In cant\_bancas : \ent}{\paraTodo[unalinea]{i}{\ent}{0 \leq i < |DHont| \yLuego |DHont[i]| > 0  \implicaLuego |DHont[i]| =  cant\_bancas}}
\pred{dHontValida}{\In DHont : \TLista{\TLista{\ent}}, \In cant\_bancas : \ent, \In escrutinio : \TLista{\ent}}{
|escrutinio| - 1 = |DHont| \land filasValidas(DHont, cant\_bancas) \land cocientesPositivos(DHont) \land cocientesDistintos(DHont)} 
\pred{generosValidos}{\In Listas : \TLista{\TLista{\ent \times \ent}}}{\paraTodo[unalinea]{i,j}{\ent}{0 \leq i < |listas| \yLuego 0 \leq j < |listas[i]| \implicaLuego listas[i][j]_{1} = 1 \lor  listas[i][j]_{1} = 2}}

\newpage

\section{Implemetaciones y demostraciones de correctitud}

\subsection{Implementación de hayBallotage}

	\begin{lstlisting}[caption={},label=code:for]
i := 0;
totalVotos := 0;
while (i < escrutinio.size()) do
	totalVotos := totalVotos + escrutinio[i];
	i := i + 1
endwhile
primero := 0;
segundo := 0;
j := 0;
while (i < escrutinio.size() - 1) do
	if (escrutinio[j] > primero) then
		segundo := primero;
		primero := escrutinio[j];
	else
		if (primero > escrutinio[j] && escrutinio[j] > segundo) then
			segundo := escrutinio[j];
		else
			skip;
		endif
	endif
	j := j + 1
endwhile
primero := (primero/totalVotos)*100
segundo :=  (segundo/totalVotos)*100
diferencia := primero - segundo
res :=  (primero <= 45 || segundo > 45) && (primero < 40 || diferencia <= 10) 
	\end{lstlisting}

\subsection{Implementación de hayFraude}

	\begin{lstlisting}[caption={},label=code:for]
i := 0;
total_presidencial := 0;
total_senadores := 0;
total_diputados := 0;
while (i < escrutinio_presidencial.size()) do
	total_presidencial := total_presidencial + escrutinio_presidencial[i];
	total_senadores := total_senadores + escrutinio_senadores[i];
	total_diputados := total_diputados + escrutinio_diputados[i];
	i := i + 1
endwhile
res := (total_presidencial = total_senadores) && (total_senadores = total_diputados)
	\end{lstlisting}

\subsection{Implementacion validarListasDiputadosEnProvincia}

	\begin{lstlisting}[caption={},label=code:for]
i := 0;
res := True
while (i < listas.size() - 1) do
	res := res && |listas[i]| = |listas[i+1]|
endwhile
while (j < listas.size()) do
	while (k < listas[j].size() - 1) do
		res := res && listas[j][k][1] != listas[j][k+1][1]
		k := k + 1
	endwhile
	j := j +1
endwhile
	\end{lstlisting}

\subsection{Implementación de obtenerSenadoresEnProvincia}

	\begin{lstlisting}[caption={},label=code:for]
primero := 0;
segundo := 0;
i := 0;
while (i < escrutinio.size() - 1) do
	if (escrutinio[i] > escrutinio[primero]) then
		segundo := primero;
		primero := i;
	else
		if (escrutinio[primero] > escrutinio[i] && escrutinio[i] > escrutinio[segundo]) then
			segundo := i;
		else
			skip;
		endif
	endif
	i := i + 1
endwhile
res := (primero, segundo)
	\end{lstlisting}


\subsection{Demostracion de correctitud de hayFraude}



Para probar que el algoritmo propuesto es correcto vamos a dividirlo en subprogramas:

\vspace{0.3cm}

$S_1$
	\begin{lstlisting}[caption={},label=code:for]
i := 0;
total_presidencial := 0;
total_senadores := 0;
total_diputados := 0;
	\end{lstlisting}

\vspace{0.3cm}

$Ciclo$
	\begin{lstlisting}[caption={},label=code:for]
while (i < escrutinio_presidencial.size()) do
	total_presidencial := total_presidencial + escrutinio_presidencial[i];
	total_senadores := total_senadores + escrutinio_senadores[i];
	total_diputados := total_diputados + escrutinio_diputados[i];
	i := i + 1
endwhile
	\end{lstlisting}

\vspace{0.3cm}

$S_2$


	\begin{lstlisting}[caption={},label=code:for]
res := (total_presidencial = total_senadores) && (total_senadores = total_diputados)
	\end{lstlisting}

\vspace{0.3cm}

Y, luego probaremos que,


\begin{enumerate}\setlength{\itemindent}{0.5cm}
		\item Requiere $\rightarrow$ wp($S_1$, $P_c$)
		\item $P_c$ $\rightarrow$ wp(Ciclo, $Q_c$)
		\item $Q_c$ $\rightarrow$ wp($S_2$, Asegura)
\end{enumerate}

Si probamos estas tres cosas, por el corolario de monotonía tenemos que, \vspace{0.2cm} \\
\indent \qquad \quad Requiere $\rightarrow$ wp(hayFraude, Asegura)

\vspace{0.3cm}

Comencemos eligiendo, \vspace{0.3cm} \\
$P_{c} \equiv i=0 \land total\_presidente = 0 \land total\_senadores = 0 \land total\_diputados = 0 \land \\ |escrutinio\_senadores| = |escrutinio\_presidente| \land |escrutinio\_senadores| = |escrutinio\_diputados| \land \\
escrutinioValido(escrutinio\_presidente)  \land escrutinioValido(escrutinio\_presidente) \land escrutinioValido(escrutinio\_presidente) \vspace{0.1cm} $
(Es lo mínimo que podemos pedir basandonos en el Requiere más aplicar S1) \vspace{0.1cm}\\
$Q_{c} \equiv total\_presidencial = totalVotos(escrutinio\_presidencial) \land  total\_diputados = totalVotos(escrutinio\_diputados) \land total\_senadores = totalVotos(escrutinio\_senadores)  \land  |escrutinio\_senadores| = |escrutinio\_presidente| \land \\ |escrutinio\_senadores| = |escrutinio\_diputados| \vspace{0.1cm} $ \\
(Similar al asegura, con la diferencia de que no hablamos de res) \vspace{0.1cm} \\
$B \equiv i < |escrutinio\_presidencial|$ \vspace{0.1cm}\\
(Es la guarda del ciclo) \vspace{0.1cm} \\
$I \equiv 0 \leq i \leq |escrutinio\_presidencial| \yLuego total\_presidencial = \sum\limits_{k=0}^{i - 1} escrutinio\_presidencial[k] \land \\
0 \leq i \leq |escrutinio\_senadores| \yLuego total\_senadores = \sum\limits_{k=0}^{i - 1} escrutinio\_senadores[k] \land \\
0 \leq i \leq |escrutinio\_diputados| \yLuego total\_diputados = \sum\limits_{k=0}^{i - 1} escrutinio\_diputados[k]$ \vspace{0.1cm} \\
(Como el ciclo incrementa i, el invariante muestra que total\_presidencial, total\_senadores y total\_diputados contienen 
las sumas parciales de sus elementos hasta i-1) \vspace{0.1cm} \\
$Fv = |escrutinio\_presidencial| - i $

\subsection{1. Requiere $\rightarrow$ wp($S_1$, $P_c$)}

\noindent Primero calculo wp($S_1$, $P_c$) \vspace{0.1cm} \\
$wp(i:=0; total\_presidencial = 0 ; total\_senadores = 0 ;total\_diputados = 0, P_c)$ \vspace{0.1cm}\\
(Por axioma 3, calculamos de afuera hacia dentro) \vspace{0.1cm}\\
$ wp(total\_diputados := 0, P_{c}) \equiv def(0) \yLuego i=0 \land total\_presidente = 0 \land total\_senadores = 0 \land 0=0\land \\ |escrutinio\_senadores| = |escrutinio\_presidente| \land |escrutinio\_senadores| = |escrutinio\_diputados| \land \\
escrutinioValido(escrutinio\_presidente)  \land escrutinioValido(escrutinio\_senadores) \land escrutinioValido(escrutinio\_diputados) \vspace{0.1cm} $ \\
$ \equiv i=0 \land total\_presidente = 0 \land total\_senadores = 0 \land |escrutinio\_senadores| = |escrutinio\_presidente| \land \\ |escrutinio\_senadores| = |escrutinio\_diputados| \land
escrutinioValido(escrutinio\_presidente)  \\ \land escrutinioValido(escrutinio\_senadores) \land escrutinioValido(escrutinio\_diputados) \equiv E_1  \vspace{0.5cm} $ 

\noindent $wp(total\_senadores := 0, E_1) \equiv def(0) \yLuego i=0 \land total\_presidente = 0 \land 0 = 0 \land |escrutinio\_senadores| = |escrutinio\_presidente| \land |escrutinio\_senadores| = |escrutinio\_diputados| \land
escrutinioValido(escrutinio\_presidente)  \\ \land escrutinioValido(escrutinio\_senadores) \land escrutinioValido(escrutinio\_diputados) \vspace{0.1cm} $

\noindent $\equiv i=0 \land total\_presidente = 0 \land |escrutinio\_senadores| = |escrutinio\_presidente| \land \\ |escrutinio\_senadores| = |escrutinio\_diputados| \land
escrutinioValido(escrutinio\_presidente)  \\ \land escrutinioValido(escrutinio\_senadores) \land escrutinioValido(escrutinio\_diputados) \equiv E_2 \vspace{0.5cm} $ 

\noindent $wp(total\_presidente := 0, E_2) \equiv def(0) \yLuego i=0 \land 0=0 \land |escrutinio\_senadores| = |escrutinio\_presidente| \\ \land |escrutinio\_senadores| = |escrutinio\_diputados| \land
escrutinioValido(escrutinio\_presidente)  \\ \land escrutinioValido(escrutinio\_senadores) \land escrutinioValido(escrutinio\_diputados) \vspace{0.1cm} $

\noindent $ \equiv i=0 \land total\_presidente = 0 \land |escrutinio\_senadores| = |escrutinio\_presidente| \land \\ |escrutinio\_senadores| = |escrutinio\_diputados| \land
escrutinioValido(escrutinio\_presidente) \land \\ escrutinioValido(escrutinio\_senadores) \land escrutinioValido(escrutinio\_diputados) \equiv E_3 \vspace{0.5cm} $

\noindent $wp(total\_presidente := 0, E_3) \equiv def(0) \yLuego 0=0 \land |escrutinio\_senadores| = |escrutinio\_presidente| \\ \land |escrutinio\_senadores| = |escrutinio\_diputados| \land
escrutinioValido(escrutinio\_presidente)  \\ \land escrutinioValido(escrutinio\_senadores) \land escrutinioValido(escrutinio\_diputados) \vspace{0.1cm} $

\noindent $ \equiv |escrutinio\_senadores| = |escrutinio\_presidente| \land |escrutinio\_senadores| = |escrutinio\_diputados| \land \\
escrutinioValido(escrutinio\_presidente) \land escrutinioValido(escrutinio\_senadores) \land \\ escrutinioValido(escrutinio\_diputados) \equiv E_4 \vspace{0.5cm} $

Ahora como $Requiere \equiv E_4, Requiere \rightarrow E_4$

\subsection{3. $Q_C$ $\rightarrow$ wp($S_2$, Asegura)}

\noindent Calculamos wp($S_2$, Asegura) \vspace{0.1cm} \\
\noindent wp($res := (total\_presidencial = total\_senadores) \land (total\_senadores = total\_diputados)$, Asegura) \vspace{0.1cm} \\
$\equiv def((total\_presidencial = total\_senadores) \land (total\_senadores = total\_diputados)) \yLuego \\ 
True = True \longleftrightarrow totalVotos(escrutinio\_presidencial) =  totalVotos(escrutinio\_senador) \land  \\  totalVotos(escrutinio\_senadores) =  totalVotos(escrutinio\_diputados \vspace{0.1cm} $

\noindent $\equiv totalVotos(escrutinio\_presidencial) =  totalVotos(escrutinio\_senador) \land  \\  totalVotos(escrutinio\_senadores) =  totalVotos(escrutinio\_diputados) \equiv E_5 \vspace{0.5cm}$

\newpage

\noindent Quiero ver que $Q_c \rightarrow E_5 \vspace{0.1cm}$

\noindent Asumo verdadero $Q_c$ y tengo que, \vspace{0.1cm}\\
totalVotos(escrutinio\_presidencial) =  totalVotos(escrutinio\_senador)\\  
totalVotos(escrutinio\_senadores) =  totalVotos(escrutinio\_diputados)

\subsection{Demostracion de correctitud de obtenerSenadoresEnProvincia}

\end{document}
