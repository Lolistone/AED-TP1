\documentclass[10pt,a4paper]{article}

\usepackage[spanish,activeacute,es-tabla]{babel}
\usepackage[utf8]{inputenc}
\usepackage{ifthen}
\usepackage{listings}
\usepackage{dsfont}
\usepackage{subcaption}
\usepackage{amsmath}
\usepackage[strict]{changepage}
\usepackage[top=1cm,bottom=2cm,left=1cm,right=1cm]{geometry}%
\usepackage{color}%
\newcommand{\tocarEspacios}{%
	\addtolength{\leftskip}{3em}%
	\setlength{\parindent}{0em}%
}

% Especificacion de procs

\newcommand{\In}{\textsf{in }}
\newcommand{\Out}{\textsf{out }}
\newcommand{\Inout}{\textsf{inout }}

\newcommand{\encabezadoDeProc}[4]{%
	% Ponemos la palabrita problema en tt
	%  \noindent%
	{\normalfont\bfseries\ttfamily proc}%
	% Ponemos el nombre del problema
	\ %
	{\normalfont\ttfamily #2}%
	\
	% Ponemos los parametros
	(#3)%
	\ifthenelse{\equal{#4}{}}{}{%
		% Por ultimo, va el tipo del resultado
		\ : #4}
}

\newenvironment{proc}[4][res]{%
	
	% El parametro 1 (opcional) es el nombre del resultado
	% El parametro 2 es el nombre del problema
	% El parametro 3 son los parametros
	% El parametro 4 es el tipo del resultado
	% Preambulo del ambiente problema
	% Tenemos que definir los comandos requiere, asegura, modifica y aux
	\newcommand{\requiere}[2][]{%
		{\normalfont\bfseries\ttfamily requiere}%
		\ifthenelse{\equal{##1}{}}{}{\ {\normalfont\ttfamily ##1} :}\ %
		\{\ensuremath{##2}\}%
		{\normalfont\bfseries\,\par}%
	}
	\newcommand{\asegura}[2][]{%
		{\normalfont\bfseries\ttfamily asegura}%
		\ifthenelse{\equal{##1}{}}{}{\ {\normalfont\ttfamily ##1} :}\
		\{\ensuremath{##2}\}%
		{\normalfont\bfseries\,\par}%
	}
	\renewcommand{\aux}[4]{%
		{\normalfont\bfseries\ttfamily aux\ }%
		{\normalfont\ttfamily ##1}%
		\ifthenelse{\equal{##2}{}}{}{\ (##2)}\ : ##3\, = \ensuremath{##4}%
		{\normalfont\bfseries\,;\par}%
	}
	\renewcommand{\pred}[3]{%
		{\normalfont\bfseries\ttfamily pred }%
		{\normalfont\ttfamily ##1}%
		\ifthenelse{\equal{##2}{}}{}{\ (##2) }%
		\{%
		\begin{adjustwidth}{+5em}{}
			\ensuremath{##3}
		\end{adjustwidth}
		\}%
		{\normalfont\bfseries\,\par}%
	}
	
	\newcommand{\res}{#1}
	\vspace{1ex}
	\noindent
	\encabezadoDeProc{#1}{#2}{#3}{#4}
	% Abrimos la llave
	\par%
	\tocarEspacios
}
{
	% Cerramos la llave
	\vspace{1ex}
}

\newcommand{\aux}[4]{%
	{\normalfont\bfseries\ttfamily\noindent aux\ }%
	{\normalfont\ttfamily #1}%
	\ifthenelse{\equal{#2}{}}{}{\ (#2)}\ : #3\, = \ensuremath{#4}%
	{\normalfont\bfseries\,;\par}%
}

\newcommand{\pred}[3]{%
	{\normalfont\bfseries\ttfamily\noindent pred }%
	{\normalfont\ttfamily #1}%
	\ifthenelse{\equal{#2}{}}{}{\ (#2) }%
	\{%
	\begin{adjustwidth}{+2em}{}
		\ensuremath{#3}
	\end{adjustwidth}
	\}%
	{\normalfont\bfseries\,\par}%
}

% Tipos

\newcommand{\nat}{\ensuremath{\mathds{N}}}
\newcommand{\ent}{\ensuremath{\mathds{Z}}}
\newcommand{\float}{\ensuremath{\mathds{R}}}
\newcommand{\bool}{\ensuremath{\mathsf{Bool}}}
\newcommand{\cha}{\ensuremath{\mathsf{Char}}}
\newcommand{\str}{\ensuremath{\mathsf{String}}}

% Logica

\newcommand{\True}{\ensuremath{\mathrm{true}}}
\newcommand{\False}{\ensuremath{\mathrm{false}}}
\newcommand{\Then}{\ensuremath{\rightarrow}}
\newcommand{\Iff}{\ensuremath{\leftrightarrow}}
\newcommand{\implica}{\ensuremath{\longrightarrow}}
\newcommand{\IfThenElse}[3]{\ensuremath{\mathsf{if}\ #1\ \mathsf{then}\ #2\ \mathsf{else}\ #3\ \mathsf{fi}}}
\newcommand{\yLuego}{\land _L}
\newcommand{\oLuego}{\lor _L}
\newcommand{\implicaLuego}{\implica _L}

\newcommand{\cuantificador}[5]{%
	\ensuremath{(#2 #3: #4)\ (%
		\ifthenelse{\equal{#1}{unalinea}}{
			#5
		}{
			$ % exiting math mode
			\begin{adjustwidth}{+2em}{}
				$#5$%
			\end{adjustwidth}%
			$ % entering math mode
		}
		)}
}

\newcommand{\existe}[4][]{%
	\cuantificador{#1}{\exists}{#2}{#3}{#4}
}
\newcommand{\paraTodo}[4][]{%
	\cuantificador{#1}{\forall}{#2}{#3}{#4}
}

%listas

\newcommand{\TLista}[1]{\ensuremath{seq \langle #1\rangle}}
\newcommand{\lvacia}{\ensuremath{[\ ]}}
\newcommand{\lv}{\ensuremath{[\ ]}}
\newcommand{\longitud}[1]{\ensuremath{|#1|}}
\newcommand{\cons}[1]{\ensuremath{\mathsf{addFirst}}(#1)}
\newcommand{\indice}[1]{\ensuremath{\mathsf{indice}}(#1)}
\newcommand{\conc}[1]{\ensuremath{\mathsf{concat}}(#1)}
\newcommand{\cab}[1]{\ensuremath{\mathsf{head}}(#1)}
\newcommand{\cola}[1]{\ensuremath{\mathsf{tail}}(#1)}
\newcommand{\sub}[1]{\ensuremath{\mathsf{subseq}}(#1)}
\newcommand{\en}[1]{\ensuremath{\mathsf{en}}(#1)}
\newcommand{\cuenta}[2]{\mathsf{cuenta}\ensuremath{(#1, #2)}}
\newcommand{\suma}[1]{\mathsf{suma}(#1)}
\newcommand{\twodots}{\ensuremath{\mathrm{..}}}
\newcommand{\masmas}{\ensuremath{++}}
\newcommand{\matriz}[1]{\TLista{\TLista{#1}}}
\newcommand{\seqchar}{\TLista{\cha}}

\renewcommand{\lstlistingname}{Código}
\lstset{% general command to set parameter(s)
	language=Java,
	morekeywords={endif, endwhile, skip},
	basewidth={0.47em,0.40em},
	columns=fixed, fontadjust, resetmargins, xrightmargin=5pt, xleftmargin=15pt,
	flexiblecolumns=false, tabsize=4, breaklines, breakatwhitespace=false, extendedchars=true,
	numbers=left, numberstyle=\tiny, stepnumber=1, numbersep=9pt,
	frame=l, framesep=3pt,
	captionpos=b,
}

\usepackage{caratula} % Version modificada para usar las macros de algo1 de ~> https://github.com/bcardiff/dc-tex


\titulo{Trabajo Práctico 1}
\subtitulo{Especificación y Weakest Precondition}

\fecha{\today}

\materia{Algoritmos y Estructuras de Datos}
\grupo{swifties}

\integrante{Apellido, Nombre1}{001/01}{email1@dominio.com}
\integrante{Apellido, Nombre2}{002/01}{email2@dominio.com}
\integrante{Apellido, Nombre3}{003/01}{email3@dominio.com}
\integrante{Apellido, Nombre4}{004/01}{email4@dominio.com}
% Pongan cuantos integrantes quieran

% Declaramos donde van a estar las figuras
% No es obligatorio, pero suele ser comodo
\graphicspath{{../static/}}

\begin{document}

\maketitle

\section{Especificación}

\noindent Algunas consideraciones que tuvimos al especificar:
\begin{enumerate} \setlength\itemsep{0cm}
	\item Un escrutinio tiene al menos dos partidos, y en la ultima posicion estan los votos en blanco.
	\item Tanto en el escrutinio como en la matriz DHont no hay repetidos.
	\item Los escrutinios en hayFraude tienen la misma longitud.
	\item Los votos en el escrutinio y los cocientes de la matriz son numeros mayores o iguales a 0.
	\item En la matriz DHont los partidos que no pasaron el umbral estaran representados como una lista vacia.
	\item En las listas de los partidos representamos a hombres con el 1 y mujeres con el 2.
\end{enumerate}
Los predicados y auxiliares utilizados están definidos al final de la sección. \\

\subsection{Ejercicio 1}

\begin{proc}{hayBallotage}{\In escrutinio : \TLista{\ent}}{Bool}
	%    \modifica{parametro1, parametro2,..}
	\requiere{escrutinioValido(escrutinio)}
	\asegura{res = True \longleftrightarrow \neg ((\paraTodo[unalinea]{j,k}{\ent}{0 \leq j,k < |escrutinio| - 1  \yLuego porcentaje(j, escrutinio) > 40  	\implicaLuego \\
		   	porcentaje(j, escrutinio) - porcentaje(k, escrutinio) > 10}) \lor \\
		  	\existe[unalinea]{i}{\ent}{\paraTodo[unalinea]{j}{\ent}{0 \leq i,j < |escrutinio| - 1 \yLuego porcentaje(i, escrutinio)  > 45 \land i \neq j  \implicaLuego \\ porcentaje(j, escrutinio) \leq 45}})}
\end{proc}

\subsection{Ejercicio 2}

\begin{proc}{hayFraude}{\In escrutinio\_presidencial : \TLista{\ent}, \In escrutinio\_senadores : \TLista{\ent}, \In escrutinio\_diputados : \TLista{\ent}}{Bool}
	%    \modifica{parametro1, parametro2,..}
	\requiere{|escrutinio\_presidencial| = |escrutinio\_senadores| \land |escrutinio\_senadores| = |escrutinio\_diputados| \land \\
	escrutinioValido(escrutinio\_presidencial) \land escrutinioValido(escrutinio\_senadores) \\ \land escrutinioValido(escrutinio\_presidencial)}
	\asegura{res = True \longleftrightarrow totalVotos(escrutinio\_presidencial) =  totalVotos(escrutinio\_senador) \land  \\  totalVotos(escrutinio\_senadores) =  totalVotos(escrutinio\_diputados)}
\end{proc}

\subsection{Ejercicio 3}

\begin{proc}{obtenerSenadoresEnProvincia}{\In escrutinio : \TLista{\ent}}{$\ent \times \ent$}
	%    \modifica{parametro1, parametro2,..}
	\requiere{escrutinioValido(escrutinio)}
	\asegura{(0 \leq res_{0}, res_{1} < |escrutinio| - 1) \yLuego \paraTodo[unalinea]{i}{\ent}{0 \leq i < |escrutinio| - 1 \land i \neq res_{0} \land 
	i \neq res_{1} \implicaLuego \\ escrutinio[i] < escrutinio[res_{1}] < escrutinio[res_{0}]}} 
\end{proc}

\subsection{Ejercicio 4}

\begin{proc}{calcularDHontEnProvincia}{\In cant\_bancas : \ent, \In escrutinio : \TLista{\ent}}{\TLista{\TLista{\ent}}}
	%    \modifica{parametro1, parametro2,..}
	\requiere{cant\_bancas > 0 \land escrutinioValido(escrutinio)}
	\asegura{|res| = |escrutinio| - 1 \yLuego \\
	\paraTodo[unalinea]{i}{\ent}{0 \leq i < |res| \yLuego superaUmbral(escrutinio[i]) \implicaLuego |res[i]| =  cant\_bancas} \land \\
	\paraTodo[unalinea]{i}{\ent}{0 \leq i < |res| \yLuego \neg(superaUmbral(escrutinio[i])) \implicaLuego |res[i]| =  0} \land \\
	\paraTodo[unalinea]{i,j}{\ent}{0 \leq i < |res| \yLuego 0 \leq j < |res[i]| \implicaLuego res[i][j] =  escrutinio[i]/(j + 1)}}
\end{proc}

\subsection{Ejercicio 5}

\begin{proc}{obtenerDiputadosEnProvincia}{\In cant\_bancas : \ent, \In escrutinio : \TLista{\ent} \In DHont : \TLista{\TLista{\ent}}}{\TLista{\ent}}
	%    \modifica{parametro1, parametro2,..}
	\requiere{cant\_bancas > 0 \yLuego dHontValida(DHont, cant\_bancas, escrutinio)}
	\asegura{|res| = |DHont| \yLuego \\ \paraTodo[unalinea]{i}{\ent}{0 \leq i < |res| \implicaLuego res[i] = \sum\limits_{j=0}^{|DHont[i]|-1} \IfThenElse{cant\_bancas > \#cocientesMayores(DHont[i][j], DHont)}{1}{0}}}
	 
\end{proc}

\subsection{Ejercicio 6}

\vspace{0.1cm}

\begin{proc}{validarListasDiputadosEnProvincia}{\In cant\_bancas : \ent, \In listas : \TLista{\TLista{dni : \ent \times genero : \ent}}}{Bool}
	%    \modifica{parametro1, parametro2,..}
	\requiere{cant\_bancas > 0 \land generosValidos(listas)}
	\asegura{res = True \longleftrightarrow \\
\paraTodo[unalinea]{i}{\ent}{0 \leq i < |listas| \implicaLuego  |listas[i]|= cant\_bancas} \land \\ 
\paraTodo[unalinea]{i,j}{\ent}{0 \leq i < |listas| \yLuego 0 \leq j < |listas[i]| - 1 \yLuego listas[i][j]_{1} = 1 \implicaLuego  listas[i][j+1]_{1} = 2 } \land \\  \paraTodo[unalinea]{i,j}{\ent}{0 \leq i < |listas| \yLuego 0 \leq j < |listas[i]| - 1 \yLuego listas[i][j]_{1} = 2 \implicaLuego  listas[i][j+1]_{1} = 1 }}
\end{proc}

\subsection{Predicados y Auxiliares}

\aux{porcentaje}{\In partido : {\ent}, \In escrutinio : \TLista{\ent}}{\ent}{(escrutinio[partido] * 100) / totalVotos(escrutinio)}
\aux{\#cocientesMayores}{\In i : \ent, \In j : \ent, \In DHont : \TLista{\TLista{\ent}}}{\ent}{\\ \sum\limits_{r=0}^{|DHont[i]|-1} \sum\limits_{k=0}^{|DHont|-1} \IfThenElse{DHont[k][r] > DHont[i][j]}{1}{0}}
\aux{totalVotos}{\In escrutinio : \TLista{\ent}}{\ent}{\sum\limits_{i=0}^{|escrutinio|-1} escrutinio[i]}
\pred{noHayEmpate}{\In escrutinio : \TLista{\ent}}{\paraTodo[unalinea]{i,j}{\ent}{0 \leq i,j < |escrutinio| - 1  \land i \neq j  \implicaLuego escrutinio[i] \neq escrutinio[j] }}
\pred{votosValidos}{\In escrutinio : \TLista{\ent}}{\paraTodo[unalinea]{i}{\ent}{0 \leq i < |escrutinio| \implicaLuego escrutinio[i] \geq 0 }}
\pred{escrutinioValido}{\In escrutinio : \TLista{\ent}}{|escrutinio| > 2 \land noHayEmpate(escrutinio) \land votosValidos(escrutinio)}
\pred{cocientesDistintos}{\In DHont : \TLista{\TLista{\ent}}}{\paraTodo[unalinea]{i,j,k,r}{\ent}{0 \leq i,k < |DHont| \yLuego 0 \leq j, r < |DHont[i]| \land
|DHont[i]| = |DHont[k]| \land (i \neq k \lor j \neq r) \implicaLuego 
DHont[i][j] \neq DHont[k][r]}}
\pred{cocientesPositivos}{\In DHont : \TLista{\TLista{\ent}}}{\paraTodo[unalinea]{i,j}{\ent}{0 \leq i < |DHont| \yLuego 0 \leq j < |DHont[i]| \implicaLuego DHont[i][j] \geq 0}}
\pred{filasValidas}{\In DHont : \TLista{\TLista{\ent}}, \In cant\_bancas : \ent}{\paraTodo[unalinea]{i}{\ent}{0 \leq i < |DHont| \yLuego |DHont[i]| > 0  \implicaLuego |DHont[i]| =  cant\_bancas}}
\pred{dHontValida}{\In DHont : \TLista{\TLista{\ent}}, \In cant\_bancas : \ent, \In escrutinio : \TLista{\ent}}{
|escrutinio| - 1 = |DHont| \land filasValidas(DHont, cant\_bancas) \land cocientesPositivos(DHont) \land cocientesDistintos(DHont)} 
\pred{generosValidos}{\In Listas : \TLista{\TLista{\ent \times \ent}}}{\paraTodo[unalinea]{i,j}{\ent}{0 \leq i < |listas| \yLuego 0 \leq j < |listas[i]| \implicaLuego listas[i][j]_{1} = 1 \lor  listas[i][j]_{1} = 2}}

\newpage

\section{Implemetaciones y demostraciones de correctitud}

\subsection{Implementación de hayBallotage}

	\begin{lstlisting}[caption={},label=code:for]
i := 0;
totalVotos := 0;
while (i < escrutinio.size()) do
	totalVotos := totalVotos + escrutinio[i];
	i := i + 1
endwhile
primero := 0;
segundo := 0;
j := 0;
while (i < escrutinio.size() - 1) do
	if (escrutinio[j] > primero) then
		segundo := primero;
		primero := escrutinio[j];
	else
		if (primero > escrutinio[j] && escrutinio[j] > segundo) then
			segundo := escrutinio[j];
		else
			skip;
		endif
	endif
	j := j + 1
endwhile
primero := (primero/totalVotos)*100
segundo :=  (segundo/totalVotos)*100
diferencia := primero - segundo
res :=  (primero <= 45 || segundo > 45) && (primero < 40 || diferencia <= 10) 
	\end{lstlisting}

\subsection{Implementación de hayFraude}

	\begin{lstlisting}[caption={},label=code:for]
i := 0;
total_presidencial := 0;
total_senadores := 0;
total_diputados := 0;
while (i < escrutinio_presidencial.size()) do
	total_presidencial := total_presidencial + escrutinio_presidencial[i];
	total_senadores := total_senadores + escrutinio_senadores[i];
	total_diputados := total_diputados + escrutinio_diputados[i];
	i := i + 1
endwhile
res := (total_presidencial = total_senadores) && (total_senadores = total_diputados)
	\end{lstlisting}

\subsection{Implementacion validarListasDiputadosEnProvincia}

	\begin{lstlisting}[caption={},label=code:for]
i := 0;
res := True
while (i < listas.size() - 1) do
	res := res && |listas[i]| = |listas[i+1]|
endwhile
while (j < listas.size()) do
	while (k < listas[j].size() - 1) do
		res := res && listas[j][k][1] != listas[j][k+1][1]
		k := k + 1
	endwhile
	j := j +1
endwhile
	\end{lstlisting}

\subsection{Implementación de obtenerSenadoresEnProvincia}

	\begin{lstlisting}[caption={},label=code:for]
primero := 0;
segundo := 0;
i := 0;
while (i < escrutinio.size() - 1) do
	if (escrutinio[i] > escrutinio[primero]) then
		segundo := primero;
		primero := i;
	else
		if (escrutinio[primero] > escrutinio[i] && escrutinio[i] > escrutinio[segundo]) then
			segundo := i;
		else
			skip;
		endif
	endif
	i := i + 1
endwhile
res := (primero, segundo)
	\end{lstlisting}


\subsection{Demostracion de correctitud de hayFraude}



\subsection{Demostracion de correctitud de obtenerSenadoresEnProvincia}

\end{document}
