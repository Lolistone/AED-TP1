\documentclass[10pt,a4paper]{article}

\input{AEDmacros}
\usepackage{caratula} % Version modificada para usar las macros de algo1 de ~> https://github.com/bcardiff/dc-tex


\titulo{Trabajo Práctico 1}
\subtitulo{Especificación y Weakest Precondition}

\fecha{\today}

\materia{Algoritmos y Estructuras de Datos}
\grupo{swifties}

\integrante{Apellido, Nombre1}{001/01}{email1@dominio.com}
\integrante{Apellido, Nombre2}{002/01}{email2@dominio.com}
\integrante{Apellido, Nombre3}{003/01}{email3@dominio.com}
\integrante{Apellido, Nombre4}{004/01}{email4@dominio.com}
% Pongan cuantos integrantes quieran

% Declaramos donde van a estar las figuras
% No es obligatorio, pero suele ser comodo
\graphicspath{{../static/}}

\begin{document}

\maketitle

\section{Especificación}

\noindent Algunas consideraciones que tuvimos al especificar:
\begin{enumerate} \setlength\itemsep{0cm}
	\item Un escrutinio tiene al menos dos partidos, y en la ultima posicion estan los votos en blanco.
	\item Tanto en el escrutinio como en la matriz DHont no hay repetidos.
	\item Los votos en el escrutinio y los cocientes de la matriz son numeros mayores o iguales a 0.
	\item En las listas de los partidos representamos a hombres con el 1 y mujeres con el 2.
\end{enumerate}
Los predicados y auxiliares utilizados están definidos al final de la sección. \\

\subsection{Ejercicio 1}

\begin{proc}{hayBallotage}{\In escrutinio : \TLista{\ent}}{Bool}
	%    \modifica{parametro1, parametro2,..}
	\requiere{escrutinioValido(escrutinio)}
	\asegura{res = True \longleftrightarrow \neg ((\paraTodo[unalinea]{j,k}{\ent}{0 \leq j,k < |escrutinio| - 1  \yLuego porcentaje(j, escrutinio) > 40  	\implicaLuego \\
		   	porcentaje(j, escrutinio) - porcentaje(k, escrutinio) > 10}) \lor \\
		  	\existe[unalinea]{i}{\ent}{\paraTodo[unalinea]{j}{\ent}{0 \leq i,j < |escrutinio| - 1 \yLuego porcentaje(i, escrutinio)  > 45 \land i \neq j  \implicaLuego \\ porcentaje(j, escrutinio) \leq 45}})}
\end{proc}

\subsection{Ejercicio 2}

\begin{proc}{hayFraude}{\In escrutinio\_presidencial : \TLista{\ent}, \In escrutinio\_senadores : \TLista{\ent}, \In escrutinio\_diputados : \TLista{\ent}}{Bool}
	%    \modifica{parametro1, parametro2,..}
	\requiere{escrutinioValido(escrutinio\_presidencial) \land escrutinioValido(escrutinio\_senadores) \land \\ escrutinioValido(escrutinio\_presidencial) }
	\asegura{res = True \longleftrightarrow totalVotos(escrutinio\_presidencial) =  totalVotos(escrutinio\_senador) \land  \\  totalVotos(escrutinio\_senadores) =  totalVotos(escrutinio\_diputados)}
\end{proc}

\subsection{Ejercicio 3}

\begin{proc}{obtenerSenadoresEnProvincia}{\In escrutinio : \TLista{\ent}}{$\ent \times \ent$}
	%    \modifica{parametro1, parametro2,..}
	\requiere{escrutinioValido(escrutinio)}
	\asegura{(0 \leq res_{0}, res_{1} < |escrutinio| - 1) \yLuego \paraTodo[unalinea]{i}{\ent}{0 \leq i < |escrutinio| - 1 \land i \neq res_{0} \land 
	i \neq res_{1} \implicaLuego \\ escrutinio[i] < escrutinio[res_{1}] < escrutinio[res_{0}]}} 
\end{proc}

\subsection{Ejercicio 4}

\begin{proc}{calcularDHontEnProvincia}{\In cant\_bancas : \ent, \In escrutinio : \TLista{\ent}}{\TLista{\TLista{\ent}}}
	%    \modifica{parametro1, parametro2,..}
	\requiere{cant\_bancas > 0 \land escrutinioValido(escrutinio)}
	\asegura{|res| = |escrutinio| - 1 \yLuego \\
	\paraTodo[unalinea]{i}{\ent}{0 \leq i < |res| \yLuego superaUmbral(escrutinio[i]) \implicaLuego |res[i]| =  cant\_bancas} \land \\
	\paraTodo[unalinea]{i}{\ent}{0 \leq i < |res| \yLuego \neg(superaUmbral(escrutinio[i])) \implicaLuego |res[i]| =  0} \land \\
	\paraTodo[unalinea]{i,j}{\ent}{0 \leq i < |res| \yLuego 0 \leq j < |res[i]| \implicaLuego res[i][j] =  escrutinio[i]/(j + 1)}}
\end{proc}

if superaUmbral then tamaño else [] endif

\subsection{Ejercicio 5}

\begin{proc}{obtenerDiputadosEnProvincia}{\In cant\_bancas : \ent, \In escrutinio : \TLista{\ent} \In DHont : \TLista{\TLista{\ent}}}{\TLista{\ent}}
	%    \modifica{parametro1, parametro2,..}
	\requiere{cant\_bancas > 0 \yLuego dHontValida(DHont, cant\_bancas)}
	\asegura{|res| = |DHont| \yLuego \\ \paraTodo[unalinea]{i}{\ent}{0 \leq i < |res| \implicaLuego res[i] = \sum\limits_{j=0}^{|		DHont[i]|-1} \IfThenElse{cant\_bancas > \#cocientesMayores(DHont[i][j], DHont)}{1}{0}}}
	 
\end{proc}


\subsection{Ejercicio 6}

\vspace{0.1cm}

\begin{proc}{validarListasDiputadosEnProvincia}{\In cant\_bancas : \ent, \In listas : \TLista{\TLista{dni : \ent \times genero : \ent}}}{Bool}
	%    \modifica{parametro1, parametro2,..}
	\requiere{cant\_bancas > 0 \land generosValidos(listas)}
	\asegura{res = True \longleftrightarrow \\
\paraTodo[unalinea]{i}{\ent}{0 \leq i < |listas| \implicaLuego  |listas[i]|= cant\_bancas} \land \\ 
\paraTodo[unalinea]{i,j}{\ent}{0 \leq i < |listas| \yLuego 0 \leq j < |listas[i]| - 1 \yLuego listas[i][j]_{1} = 1 \implicaLuego  listas[i][j+1]_{1} = 2 } \land \\  \paraTodo[unalinea]{i,j}{\ent}{0 \leq i < |listas| \yLuego 0 \leq j < |listas[i]| - 1 \yLuego listas[i][j]_{1} = 2 \implicaLuego  listas[i][j+1]_{1} = 1 }}
\end{proc}

\subsection{Predicados y Auxiliares}

\aux{porcentaje}{\In partido : {\ent}, \In escrutinio : \TLista{\ent}}{\ent}{(escrutinio[partido] * 100) / totalVotos(escrutinio)}
\aux{\#cocientesMayores}{\In i : \ent, \In j : \ent, \In DHont : \TLista{\TLista{\ent}}}{\ent}{\\ \sum\limits_{r=0}^{|DHont[i]|-1} \sum\limits_{k=0}^{|DHont|-1} \IfThenElse{DHont[k][r] > DHont[i][j]}{1}{0}}
\aux{totalVotos}{\In escrutinio : \TLista{\ent}}{\ent}{\sum\limits_{i=0}^{|escrutinio|-1} escrutinio[i]}
\pred{noHayEmpate}{\In escrutinio : \TLista{\ent}}{\paraTodo[unalinea]{i,j}{\ent}{0 \leq i,j < |escrutinio| - 1  \land i \neq j  \implicaLuego escrutinio[i] \neq escrutinio[j] }}
\pred{votosValidos}{\In escrutinio : \TLista{\ent}}{\paraTodo[unalinea]{i}{\ent}{0 \leq i < |escrutinio| \implicaLuego escrutinio[i] \geq 0 }}
\pred{escrutinioValido}{\In escrutinio : \TLista{\ent}}{|escrutinio| > 2 \land noHayEmpate(escrutinio) \land votosValidos(escrutinio)}
\pred{cocientesDistintos}{\In DHont : \TLista{\TLista{\ent}}}{\paraTodo[unalinea]{i,j,k,r}{\ent}{0 \leq i,k < |DHont| \yLuego 0 \leq j, r < |DHont[i]| \yLuego
|DHont[i]| = |DHont[k]| \yLuego (i \neq k \lor j \neq r) \implicaLuego 
DHont[i][j] \neq DHont[k][r]}}
\pred{cocientesPositivos}{\In DHont : \TLista{\TLista{\ent}}}{\paraTodo[unalinea]{i,j}{\ent}{0 \leq i < |DHont| \yLuego 0 \leq j < |DHont[i]| \implicaLuego DHont[i][j] \geq 0}}
\pred{filasValidas}{\In DHont : \TLista{\TLista{\ent}}, \In cant\_bancas : \ent}{\paraTodo[unalinea]{i}{\ent}{0 \leq i < |DHont| \yLuego |DHont[i]| > 0  \implicaLuego |DHont[i]| =  cant\_bancas}}
\pred{dHontValida}{\In DHont : \TLista{\TLista{\ent}}, \In cant\_bancas : \ent}{filasValidas(DHont, cant\_bancas) \yLuego cocientesPositivos(DHont) \yLuego cocientesDistintos(DHont)} 
\pred{generosValidos}{\In Listas : \TLista{\TLista{\ent \times \ent}}}{\paraTodo[unalinea]{i,j}{\ent}{0 \leq i < |listas| \yLuego 0 \leq j < |listas[i]| \implicaLuego listas[i][j]_{1} = 1 \lor  listas[i][j]_{1} = 2}}

\newpage

\section{Implemetaciones y demostraciones de correctitud}

\subsection{Implementación de hayBallotage}

	\begin{lstlisting}[caption={},label=code:for]
i := 0;
totalVotos := 0;
while (i < escrutinio.size()) do
	totalVotos := totalVotos + escrutinio[i];
	i := i + 1
endwhile
primero := 0;
segundo := 0;
j := 0;
while (i < escrutinio.size() - 1) do
	if (escrutinio[j] > primero) then
		segundo := primero;
		primero := escrutinio[j];
	else
		if (primero > escrutinio[j] && escrutinio[j] > segundo) then
			segundo := escrutinio[j];
		else
			skip;
		endif
	endif
	j := j + 1
endwhile
primero := (primero/totalVotos)*100
segundo :=  (segundo/totalVotos)*100
diferencia := primero - segundo
res :=  (primero <= 45 || segundo > 45) && (primero < 40 || diferencia <= 10) 
	\end{lstlisting}

\subsection{Implementación de hayFraude}

	\begin{lstlisting}[caption={},label=code:for]
i := 0;
total_presidencial := 0;
while (i < escrutinio_presidencial.size()) do
	total_presidencial := total_presidencial + escrutinio_presidencial[i];
	i := i + 1
endwhile
j := 0;
total_senadores := 0;
while (j < escrutinio_senadores.size()) do
	total_senadores := total_senadores + escrutinio_senadores[i];
	j := j + 1
endwhile
k := 0;
total_diputados := 0;
while (k < escrutinio_diputados.size()) do
	total_diputados := total_diputados + total_diputados[i];
	k := k + 1
endwhile
res := (total_presidencial = total_senadores) && (total_senadores = total_diputados)
	\end{lstlisting}

\newpage

\subsection{Implementación de obtenerSenadoresEnProvincia}

	\begin{lstlisting}[caption={},label=code:for]
primero := 0;
segundo := 0;
i := 0;
while (i < escrutinio.size() - 1) do
	if (escrutinio[i] > escrutinio[primero]) then
		segundo := primero;
		primero := i;
	else
		if (escrutinio[primero] > escrutinio[i] && escrutinio[i] > escrutinio[segundo]) then
			segundo := i;
		else
			skip;
		endif
	endif
	i := i + 1
endwhile
res := (primero, segundo)
	\end{lstlisting}

\subsection{Implementacion validarListasDiputadosEnProvincia}

	\begin{lstlisting}[caption={},label=code:for]
i := 0;
res := True
while (i < listas.size() - 1) do
	res := res && |listas[i]| = |listas[i+1]|
endwhile
while (j < listas.size()) do
	while (k < listas[j].size() - 1) do
		res := res && listas[j][k][1] != listas[j][k+1][1]
		k := k + 1
	endwhile
	j := j +1
endwhile
	\end{lstlisting}

\subsection{Demostracion de correctitud de hayFraude}

\subsection{Demostracion de correctitud de obtenerSenadoresEnProvincia}

\end{document}
